% !TeX spellcheck = pt_BR

%%%%%%%%%%%%%%%%%%%%%%%%%%%%%%%%%%%%%%%%
% Classe do documento
%%%%%%%%%%%%%%%%%%%%%%%%%%%%%%%%%%%%%%%%

% Opções:
%  - Graduação: bacharelado|engenharia|licenciatura
%  - Pós-graduação: [qualificacao], mestrado|doutorado, ppca|ppginf

% \documentclass[engenharia]{UnB-CIC}%
\documentclass[qualificacao,bacharelado,ppca]{UnB-CIC}%

\usepackage{pdfpages}% incluir PDFs, usado no apêndice

%%%%%%%%%%%%%%%%%%%%%%%%%%%%%%%%%%%%%%%%
% Informações do Trabalho
%%%%%%%%%%%%%%%%%%%%%%%%%%%%%%%%%%%%%%%%
\orientador{\prof \dr Guilherme Novaes Ramos}{CIC/UnB}%
%\coorientador{\prof \dr José Ralha}{CIC/UnB}
\coordenador[a]{\prof[a] \dr[a] Ada Lovelace}{Bibliothèque universelle de Genève}%
\diamesano{24}{dezembro}{2014}%

\membrobanca{\prof \dr Donald Knuth}{Stanford University}%
\membrobanca{\dr Leslie Lamport}{Microsoft Research}%

\autor{Guilherme N.}{Ramos}%

\titulo{UnB-CIC: Uma classe em LaTeX para textos do Departamento de Ciência da Computação}%

\palavraschave{Redes Bayesianas, UnBBayes, Aprendizado, Design Pattern}
\keywords{\LaTeX, scientific method}

\newcommand{\unbcic}{\texttt{UnB-CIC}}%






%%%%%%%%%%%%%%%%%%%%%%%%%%%%%%%%%%%%%%%%
% TextoF
%%%%%%%%%%%%%%%%%%%%%%%%%%%%%%%%%%%%%%%%



\begin{document}
  \maketitle

  \begin{dedicatoria}
Citando o poeta: ``Eu dedico essa música a primeira garota que tá sentada ali na fila. Brigado!''
  \end{dedicatoria}

  \begin{agradecimentos}
Agradeço ao Prof. José Ralha, cujos esforços na versão anterior foram bem ``adaptados'' a este trabalho.
  \end{agradecimentos}

\hyphenation{au-xi-li-ar}

  \begin{resumo}
	Com o crescente interesse em aprendizado de máquina, frameworks robustos e implementações do estado da arte são uma necessidade do mundo moderno. Pensando nisso levantamos um estudo sobre o estado da arte de aprendizado no contexto de \gls{bn}, já que \glspl{bn} provêm um excelente modelo para lidar com relações causais e probabilidade bayesiana. Para tal utilizaremos o framework UnBBayes e desenvolveremos plugins para este.
  \end{resumo}
  

  \selectlanguage{american}
  \begin{abstract}
  With the growing interest in machine learning, good frameworks and state of the art implementation are a need in modern world. With this in mind we provide a study in the state of the art of \gls{bn}, since \glspl{bn} provide a great model to deal with causal relations and bayesian probability. For such we shall use UnBBayes framework and implement plugins for it.
  \end{abstract}
  \selectlanguage{brazil}
\hyphenation{a-tri-bu-tos}

  \tableofcontents
  \listoffigures
  \listoftables
  \printnoidxglossaries
\renewcommand{\appendixname}{Anexo}


  \textual
  
\gls{bn} é um modelo gráfico para relações probabilísticas dado conjunto de variáveis. Nas últimas décadas, redes Bayesianas se tornaram representações populares para codificar conhecimento especialista incerto para sistemas especialistas \cite{heck95}. Mais recentemente, pesquisadores desenvolveram métodos de aprendizagem de redes Bayesianas a partir de dados. As técnicas desenvolvidas são novas e ainda em evolução, mas eles têm se mostrado muito eficientes para alguns problemas de análise de dados.

Existem diversos representações possíveis para analise de dados, entre elas, rule bases, decision tres, e redes neurais artificiais; e outras tantas como estimação de densidade, classificação, regressão e clusterings. Portanto o que métodos de \glspl{bn} têm a oferecer? Segundo Heckerman \cite{heck95} podemos oferecer pelo menos quatro respostas, sendo elas:

\begin{enumerate}
	\item \glspl{bn} lidam com um conjunto incompleto de dados de maneira natural.

	\item \glspl{bn} permitem aprender sobre as relações causais. Aprender sobre tais relações são importantes por pelo menos duas razões: O processo é útil quando se está tentando entender sobre um dado problema de domínio, como por exemplo, durante uma análise de dados exploratória.  E mais, conhecimento de relações causais nos permitem fazer predições na presença de intervenções. Por exemplo, um analista de mercado pode querer saber se é lucrativo aumentar o investimento em determinada propaganda para aumentar as vendas de seu produto. Para responder esta pergunta o analista pode determinar se esta propaganda é a causa para o aumento de suas vendas, e em caso afirmativo, quanto. O uso de \glspl{bn} nos ajuda a responder tal pergunta até mesmo quando não há experimentos nos efeitos de tal propaganda.
	
	\item \glspl{bn} em conjunto com técnicas estatísticas bayesianas facilitam a combinação de conhecimento de domínio e dados. Qualquer um que tenha feito uma análise do mundo real sabe a importância de conhecimento prévio ou de domínio, em especial quando os dados são poucos ou caros. Pelo fato de alguns sistemas comerciais (i.e., sistemas especialistas) podem ser construídos a partir de conhecimentos prévios. \cglspl{bn} possuem uma semântica causal que permitem conhecimentos prévios serem representados de uma forma muito simples e natural. Além disto, \glspl{bn} encapsulam tais relações causais com suas probabilidades. Consequentemente, conhecimento prévio e dados podem ser combinados com técnicas bem estudadas da estatística Bayesiana.
	
	\item Métodos Bayesianos em conjunto com \glspl{bn} e outros tipos de modelos oferecem uma forma eficiente para evitar over fitting dos dados. Como veremos, não há necessidade de excluir parte dos dados do treinamento do aprendizado da rede. Usando técnicas Bayesianas, modelos podem ser "suavizados" de tal forma que todo dado disponível pode ser usado para o treinamento.

\end{enumerate}
\section{Probabilidade física versus probabilidade Bayesiana}
Para entender \glspl{bn} e as técnicas de aprendizado associadas, é importante entender a diferença entre a Probabilidade e Estatística padrão e a Bayesiana.

Resumidamente a probabilidade e padrão, também cohecida como probabilidade física é aquela que mede a probabilidade real do mundo. Como por exemplo o lançamento de moedas. É bem aceito que há 50\% de probabilidade de um lançamento de uma moeda não viciada dê cara, pois temos duas possibilidades e todas elas são igualmente prováveis. Outro exemplo é a probabilidade de tirar uma carta de paus dado um baralho padrão, dado que o baralho está embaralhado e que contém 13 cartas de cada naipe, a probabilidade da carta do topo do baralho ser de paus é 25\%. Isto é a probabilidade física, aquela probabilidade que todos estamos acostumados a lidar desde sempre.

No entanto, imagine agora que queremos determinar a probabilidade de tirar 

  % !TeX spellcheck = pt_BR
\chapter{Redes Bayesianas}%
Uma \gls{bn} provê uma representação compacta de distribuições de probabilidades grandes demais para lidar usando especificações tradicionais e provê um método sistemático e localizado para incorporar informação probabilística sobre uma dada situação.

Uma BN é um grafo acíclico direcionado (DAG) que representa uma função de distribuição de probabilidades conjunta de variáveis que modelam certo domínio de conhecimento. Ela é constituída de uma DAG, de variáveis aleatórias (também chamadas de nós da rede), arcos direcionados da variável pai para a variável filha e uma tabela de probabilidades condicionais (CPT) associada a cada variável.
\begin{figure}[H]
	\centering
	\includegraphics[width = 300px]{figuras/BN1}
	\caption[extraída do artigo do Laecio]{Exemplo family-out}
	\label{fig:familyBN}
\end{figure}

Nesse exemplo, suponhamos que se queira determinar se a família está em casa ou se ela saiu. Pelo grafo, pdoemos perceber que o fato de a luz da varanda estar acesa e de o cachorro estar fora de casa são indícios de que a família tenha saído. 

\section{Definição Formal}
Uma rede Bayesiana consiste em uma fatoração de uma distribuição de probabilidade e um DAG correspondente. Tais assertivas de independências condicionais podem ser inferidas diretamente da fatoração correspondente às 
  %\chapter{Aprendizado de Redes Bayesianas}
Muitas vezes, quando queremos construir uma \gls{bn}, o conhecimento do relacionamento de causa entre as variáveis do nosso dominio pode ser incerto, o custo de um especialista muito elevado, e principalmente, precisar as probabilidades dos de cada nó dados seus pode ser inviável. No entanto, se tivermos dados sobre o problema, isto pode facilitar muito esse processo de criação da \gls{bn}, tudo que precisamos fazer é adaptar técnicas de aprendizado de máquina para o escopo de \glspl{bn}. Pensando nisto o interesse em desenvolver e implementar estas técnicas vem aumentando nos últimos anos.

É importante observar que as probabilidades representadas por \gls{bn} pode ser Bayesiana ou Física. Quando construímos uma \gls{bn} a partir de conhecimento prévio tão somente, as probabilidades serão Bayesianas. No entanto se aprendermos estas estruturas a partir de dados, estas probabilidades serão físicas \cite{heck95}.

O aprendizado de uma \gls{bn} é dividido em duas etapas distintas e independentes
\begin{itemize}
	\item O Aprendizado da Estrutura da rede. Isto é, quais as relações de causas entre as variáveis do nosso domínio.
	\item O Aprendizado dos Parâmetros da rede. Isto é, dada a estrutura da rede, quais as probabilidades de cada um dos nós.
\end{itemize}

Obviamente para se aprender os parâmetros de uma rede é necessário que já se tenha a estrutura da pronta. Esta estrutura pode ter sido construída por um especialista, ou aprendida pelos dados. Entretanto o aprendizado de parâmetro é trivial de ser feito e por este motivo já se tornou uma tradição entre as publicações sobre de Aprendizado de Redes Bayesianas apresentar a aprendizagem de parâmetros antes da aprendizagem de estrutura. Nós também seguiremos esta tradição.

\section{Aprendizado de Parâmetros}
O aprendizado de parâmetros nada mais que encontrar a distribuição de probabilidade conjunta de cada variável aleatória presente na reade, representadas por \glspl{cpt}, dada a topologia da rede.

Seja $\textbf{X} = {X_1,X_2,...,X_n}$ para o conjunto de variáveis aleatórias do model, $B(S)$ para a estrutura da \gls{bn} e $\theta_s$ para os parâmetros da \gls{bn}. Pearl \cite{pearl88} provou que a função de distribuição conjunta de $\textbf{X}$ pode ser obitida como o produto das distribuições de probabilidades condicionais da variável da \gls{bn}, dado os seus pais. A partir da \autoref{eq:pearl} temos que:
\begin{equation}
	P(\textbf{X}|\theta_s, B(S)) = \prod_{i=1}^{n}p(x_i|pa_i,\theta_i,B(S))
\end{equation}
Onde $\theta_i$ é o vetor de prâmetros para $P(x_i|pa_i, \theta_i, B(S))$, $\theta_s$ é o vetor de parâmetros de $(\theta_1,...,\theta_n)$. O que desejamos é econtrar os parâmetros $\theta_s$ dado um conjunto de treinamento $D$ e a estrutura $B(S)$. Para isto avaliamos a expressão $P(\theta_s| D,B(S))$. É importante notar que $D$ deve ser um conjunto de terinamento completo e é respresentado por ${x_1,x_2,...,x_n}$, onde cada $x_i$ representa um caso do conjunto de dados observados. As incertezas são codificadas sobre os parâmetros $\theta_s$ por uma variável aleatória $\Theta_s$ e a função a priori $P(\theta_s|B(S))$. A função $P(x_i|pa_i,\theta_i,B(S))$ é vista como a função de distribuição local.

\iffalse
	Assumindo que se trata de dados discretos a função de distribuição local será dada por:
	\begin{equation}
	p(x_{i}^{k}|pa_{i}^{j},\theta_i, B(S)) = \theta_{ijk} > 0
	\end{equation}
	onde $pa_{i}^1,...,pa_{i}^{q_i}$são as $q_i$ instâncias possíveis dos pais de $x_i$, $\theta_i=((\theta_{ijk})_{k=1}^{r_i})_{j=1}^{q_i}$ e $\theta_{ijk}$ é a probabilidade de ocorrência do k-ésimo valor de $x_i$, dada a j-ésima configuração de pais de $x_i$, o vetor de parâmetros $\theta_i$ dessa distribuição de probabilidades e a estrutura $B(S)$.
\fi

A partir de manipulações aritméticas que fogem do escopo deste trabalho chegamos na seguinte equação:
\begin{equation}
P(X_i=x_i^k|Pais = pa_i^j) = \frac{\alpha_{ijk}+N_{ijk}}{\sum_{i}^{r_i}\alpha_{ijk}+N_{ij}}
\label{eq:param_learn}
\end{equation}

Seja o domínio de $X_i$ denotado por $D_{x_i}$
\begin{itemize}
	\item $X_i$ é a i-ésima variável da rede bayesiana;
	\item $x_i^k \in D_{x_i}$ é a k-ésima instância da variável $X_i$;
	\item $pa_i^j$ é a j-ésima instância da variável $X_i$;
	\item $\alpha_{ijk}$ é o parâmetro da distribuição de Dirichlet;
	\item $r_i$ é a cardinalidade dos estados da variável $X_i$, de $D_{x_i}$;
	\item $N_{ij}$ é o número total de ocorrências de $X_i$ dados os seus pais; $pa_i^j$, isto é, $N_{ij}=\sum_{k=1}^{n}N_{ijk}$
\end{itemize}

Os parâmetros $\alpha_{ijk}$ podem ser substituídos por 1, pois corresponde ao valor esperado da frequência de cada estado, admitindo-se uma distribuição uniforme para os estados de $X_i$, visto que, a princípio, não se tem nenhuma informação que permita estimação melhor para essa distribuição de probabilidades.

A seguir daremos um exemplo da aplicação dessa fórmula para o aprendizado da rede Ásia \ref{fig:red_asia} proposto por Lauritzen e Spielgelhater \cite{lauritzen88}. A tabela com os dados foi retirada de \cite{custodio05}


\begin{figure}[ht]
	\centering
	\includegraphics{figuras/red_asia}
	\caption[Rede Ásia]{Rede Ásia}
	\label{fig:red_asia}
\end{figure}

\begin{figure}[ht]
	\centering
	\includegraphics{figuras/tabela_asia}
	\caption[Dados para Treinamento para a rede Ásia]{Exemplo de conjunto de treinamento para a Rede Ásia}
	\label{fig:tabela_asia}
\end{figure}

\newtheorem{exmp}{Exemplo}[chapter]
\begin{exmp}
	Neste exemplo desejamos aprender os parâmetros para a variável dispnéia da rede Àsia, a partir do conjunto de treinamento apresentado na Tabela \ref{fig:tabela_asia}
\end{exmp}


Os pais da variável da variável Dispneia são TouC (coluna 4) e Bronquite (coluna 6), portanto $pa_7=[4,6]$. Os estados de TouC e Bronquite são, respectivamente representados por:
\begin{itemize}
	\item $D_x[4]=[0,1]$, onde 0 representa 'Não' e 1 'Sim'; 
	\item $D_x[6]=[0,1]$, onde 0 representa 'Ausente' e 1 representa 'Presente'.
\end{itemize} 
As possíveis instâncias para Dispneia são: $pa_7 = [[0,0]^0, [0,1]^1,[1,0]^2,[1,1]^3]$ onde por exemplo a instância 2: $[1,0]^2$ representa TouC assumindo o valor 'Sim ' e Bronquite assumindo o valor 'Ausente'.

A matriz $N_{ijk}$ para dispnéia, obtida através da contagem na matriz D das ocorrências de Dispneia, condicionadas as instâncias possíveis para os pais TouC e Bronquite, está apresentada na \refTab{tab:disp}

\tabela{Matriz $N_{ijk}$ Para Dispneia com Pais T ou C e Bronquite}{tab:disp}{| c | c | c | c | c |}%
{\hline
	& \multicolumn{4}{|c|}{Instancia dos Pais} \\\hline
	Dispneia & 0 & 1 & 2 & 3 \\\hline
	0 & 3 & 2 & 1 & 0 \\\hline
	1 & 1 & 7 & 0 & 0 \\\hline
	}%
A partir dessa taela é possível calcular a probabilidade associada a cada parâmetro da distribuição de probabilidade da variável aleatória DIspneia. A tabela de probabiliades condicionais de Dispneia, calculada com a \ref{eq:param_learn}, está representada na \refTab{tab:cpt_disp}

\tabela{CPT calculada para Dispneia}{tab:cpt_disp}{| c | c | c | c | c | c |}%
{\hline
	& TouC & 0 & 0 & 0 & 1 \\\hline
	& Bronquite & 0 & 1 & 0 & 1\\\hline\rule[-2.5ex]{0pt}{7ex}
	\multirow{2}{*}{Dispneia} & Sim & $\frac{1+3}{2+4} = 0.67$ & $\frac{1+2}{2+9} = 0.27$ & $\frac{1+1}{2+1} = 0.67$ & $\frac{1+0}{2+0} = 0.5$ \\\cline{2-6}\rule[-2.5ex]{0pt}{7ex}
	& Não & $\frac{1+1}{2+4}=0.33$ & $\frac{1+7}{2+9}=0.73$ & $\frac{1+0}{2+1}=0.33$ & $\frac{1+0}{2+0}=0.5$\\\hline
	
	
	
}%

\section{Aprendizado de Estrutura}
O Objetivo da aprendizagem da Estrutura é encontrar a topologia da \gls{bn} que mais se adequa aos nossos dados. Para isto podemos atacar o problema de duas formas distintas:
\begin{itemize}
	\item Busca e pontuação: fazemos uma busca no conjunto de todos \gls{dag} existentes entre nossas variáveis usando heurísticas robustas o suficiente.
	\item Análise de dependência: utilizamos técnicas estatísticas bem desenvolvidas para analisar a dependência entre nossos dados e a partir deles inferir a estrutura da rede.
\end{itemize}

Vamos começar discutindo sobre algoritmos de busca e pontuação, apresentamos as heurísticas mais famosas.

\subsection{Busca e Pontuação}



\subsection{Análise de Dependência}

\section{Aprendizado Incremental}
  %\chapter{UnBBayes}%


\section{Plugins}

\subsection{Como desenvolver plugins}
  %\chapter{Design Patterns}%


\section{Padrões de Comportamento}
\section{Padrões de Estrutura}
\section{Padrões de Construção}

  \input{tex/Arquitetura}
  \section{O que foi feito}

\section{O que será feito}
  \input{tex/Conclusao}
  % inserir demais capítulos
  

  \postextual
  \bibliographystyle{plain}
  \bibliography{bibliografia}

\appendix
  %\begin{figure}[H]
	\centering
	\includegraphics[width = 200px]{figuras/singleton_pattern_uml_diagram}
	\caption {Diagrama UML do Design Pattern Singleton}
	\label{fig:dp_singleton}
\end{figure}
\begin{figure}[H]
	\centering
	\includegraphics[width = 350px]{figuras/factory_pattern_uml_diagram}
	\caption {Diagrama UML do Design Pattern Factory}
	\label{fig:dp_factory}
\end{figure}
\begin{figure}[H]
	\centering
	\includegraphics[width = 350px]{figuras/builder_pattern_uml_diagram}
	\caption {Diagrama UML do Design Pattern Builder}
	\label{fig:dp_builder}
\end{figure}
\begin{figure}[H]
	\centering
	\includegraphics[width = 350px]{figuras/prototype_pattern_uml_diagram}
	\caption {Diagrama UML do Design Pattern Prototype}
	\label{fig:dp_prototype}
\end{figure}


\begin{figure}[H]
	\centering
	\includegraphics[width = 350px]{figuras/adapter_pattern_uml_diagram}
	\caption {Diagrama UML do Design Pattern Adapter}
	\label{fig:dp_adapter}
\end{figure}
\begin{figure}[H]
	\centering
	\includegraphics[width = 350px]{figuras/bridge_pattern_uml_diagram}
	\caption {Diagrama UML do Design Pattern Bridge}
	\label{fig:dp_bridge}
\end{figure}
\begin{figure}[H]
	\centering
	\includegraphics[width = 200px]{figuras/composite_pattern_uml_diagram}
	\caption {Diagrama UML do Design Pattern Composite}
	\label{fig:dp_composite}
\end{figure}
\begin{figure}[H]
	\centering
	\includegraphics[width = 350px]{figuras/decorator_pattern_uml_diagram}
	\caption {Diagrama UML do Design Pattern Decorator}
	\label{fig:dp_decorator}
\end{figure}
\begin{figure}[H]
	\centering
	\includegraphics[width = 350px]{figuras/facade_pattern_uml_diagram}
	\caption {Diagrama UML do Design Pattern Facade}
	\label{fig:dp_facade}
\end{figure}
\begin{figure}[H]
	\centering
	\includegraphics[width = 350px]{figuras/proxy_pattern_uml_diagram}
	\caption {Diagrama UML do Design Pattern Proxy}
	\label{fig:dp_proxy}
\end{figure}


\begin{figure}[H]
	\centering
	\includegraphics[width = 350px]{figuras/chain_pattern_uml_diagram}
	\caption {Diagrama UML do Design Pattern Responsability}
	\label{fig:dp_responsability}
\end{figure}
\begin{figure}[H]
	\centering
	\includegraphics[width = 350px]{figuras/command_pattern_uml_diagram}
	\caption {Diagrama UML do Design Pattern Command}
	\label{fig:dp_command}
\end{figure}
\begin{figure}[H]
	\centering
	\includegraphics[width = 350px]{figuras/interpreter_pattern_uml_diagram}
	\caption {Diagrama UML do Design Pattern Interpreter}
	\label{fig:dp_interpreter}
\end{figure}
\begin{figure}[H]
	\centering
	\includegraphics[width = 350px]{figuras/iterator_pattern_uml_diagram}
	\caption {Diagrama UML do Design Pattern Iterator}
	\label{fig:dp_iterator}
\end{figure}
\begin{figure}[H]
	\centering
	\includegraphics[width = 350px]{figuras/mediator_pattern_uml_diagram}
	\caption {Diagrama UML do Design Pattern Mediator}
	\label{fig:dp_mediator}
\end{figure}
\begin{figure}[H]
	\centering
	\includegraphics[width = 350px]{figuras/null_pattern_uml_diagram}
	\caption {Diagrama UML do Design Pattern Null Object}
	\label{fig:dp_null_object}
\end{figure}


\begin{figure}[H]
	\centering
	\includegraphics[width = 350px]{figuras/observer_pattern_uml_diagram}
	\caption {Diagrama UML do Design Pattern Observer}
	\label{fig:dp_observer}
\end{figure}
\begin{figure}[H]
	\centering
	\includegraphics[width = 350px]{figuras/state_pattern_uml_diagram}
	\caption {Diagrama UML do Design Pattern State}
	\label{fig:dp_state}
\end{figure}

\begin{figure}[H]
	\centering
	\includegraphics[width = 350px]{figuras/template_pattern_uml_diagram}
	\caption {Diagrama UML do Design Pattern Template}
	\label{fig:dp_template}
\end{figure}
\begin{figure}[H]
	\centering
	\includegraphics[width = 350px]{figuras/visitor_pattern_uml_diagram}
	\caption {Diagrama UML do Design Pattern Visitor}
	\label{fig:dp_visitor}
\end{figure}



\end{document}
