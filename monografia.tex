% !TeX spellcheck = pt_BR

%%%%%%%%%%%%%%%%%%%%%%%%%%%%%%%%%%%%%%%%
% Classe do documento
%%%%%%%%%%%%%%%%%%%%%%%%%%%%%%%%%%%%%%%%

% Nós usamos a classe "unb-cic".  Deixe apenas uma das linhas
% abaixo não-comentada, dependendo se você for do bacharelado ou
% da licenciatura.

% Para tirar os comentários, é só mudar o comando para fazer nada.
\newcommand{\com}[1]{\textcolor{red}{#1}}%

\documentclass[mpca]{unb-cic}



%%%%%%%%%%%%%%%%%%%%%%%%%%%%%%%%%%%%%%%%
% Pacotes importados
%%%%%%%%%%%%%%%%%%%%%%%%%%%%%%%%%%%%%%%%

\usepackage[brazil,american]{babel}
\usepackage[T1]{fontenc}
\usepackage{indentfirst}
\usepackage{natbib}
\usepackage{xcolor,graphicx,url}
\usepackage[utf8]{inputenc}
\usepackage{amsmath,amssymb,amsthm}
\usepackage{footnote}
%\usepackage{minipage}
\usepackage{float}
\usepackage{tablefootnote} 
\usepackage{listings}
\usepackage{myglossary}


%%%%%%%%%%%%%%%%%%%%%%%%%%%%%%%%%%%%%%%%
% Cores dos links
%%%%%%%%%%%%%%%%%%%%%%%%%%%%%%%%%%%%%%%%

% Veja o arquivos cores.tex se quiser ver que outras cores estão
% pré-definidas.  Utilizando o comando \hypersetup abaixo nós
% evitamos aquelas caixas vermelhas feias em volta dos links.

\input{cores}
\hypersetup{
  colorlinks=true,
  linkcolor=DarkScarletRed,
  citecolor=DarkScarletRed,
  filecolor=DarkScarletRed,
  urlcolor= DarkScarletRed
}



%%%%%%%%%%%%%%%%%%%%%%%%%%%%%%%%%%%%%%%%
% Informações sobre a monografia
%%%%%%%%%%%%%%%%%%%%%%%%%%%%%%%%%%%%%%%%
\title{UnBBayes- Um Estudo}%

\orientador{\prof \dr Marcelo Ladeira}{CIC/UnB}
\coorientador{\prof Shou Matsumoto}{CIC/GMU}
\coordenador[a]{\prof[a] \dr[a] Coorde Nadora}{CIC/UnB}
\diamesano{07}{dezembro}{2015}%

\membrobanca{\prof[a] \dr[a] Membra da Banca}{MEC}
\membrobanca{\prof \dr Membro do Banco}{CIC/UnB}

\autor{Guilherme C.}{Torres}
\autor{Guilherme N.}{Ramos}
\CDU{004.4}

\palavraschave{\LaTeX, Redes Bayesianas, UnBBayes}
\keywords{\LaTeX, scientific method}



\graphicspath{{.}{img/}}%
\newcommand{\unbcic}{\texttt{UnB-CIC}}%





%%%%%%%%%%%%%%%%%%%%%%%%%%%%%%%%%%%%%%%%
% TextoF
%%%%%%%%%%%%%%%%%%%%%%%%%%%%%%%%%%%%%%%%



\begin{document}
  \maketitle

  \begin{dedicatoria}
Citando o poeta: ``Eu dedico essa música a primeira garota que tá sentada ali na fila. Brigado!''
  \end{dedicatoria}

  \begin{agradecimentos}
Agradeço ao Prof. José Ralha, cujos esforços na versão anterior foram bem ``adaptados'' a este trabalho.
  \end{agradecimentos}

\hyphenation{au-xi-li-ar}

  \begin{resumo}
  Um parágrafo resumindo todo o trabalho. Classe \LaTeX\ para gerar documentos do Departamento de Ciência da Computação da Universidade de Brasília.
  \end{resumo}
  

  \selectlanguage{american}
  \begin{abstract}
  Um parágrafo \emph{na língua Inglesa} resumindo todo o trabalho. A \LaTeX\ class for generating documents for the Departament of Computer Science of the University of Brasília.
  \end{abstract}
  \selectlanguage{brazil}
\hyphenation{a-tri-bu-tos}

  \tableofcontents
  \listoffigures
  \listoftables
  \printnoidxglossaries
\renewcommand{\appendixname}{Anexo}


  \textual
  
\gls{bn} é um modelo gráfico para relações probabilísticas dado conjunto de variáveis. Nas últimas décadas, redes Bayesianas se tornaram representações populares para codificar conhecimento especialista incerto para sistemas especialistas \cite{heck95}. Mais recentemente, pesquisadores desenvolveram métodos de aprendizagem de redes Bayesianas a partir de dados. As técnicas desenvolvidas são novas e ainda em evolução, mas eles têm se mostrado muito eficientes para alguns problemas de análise de dados.

Existem diversos representações possíveis para analise de dados, entre elas, rule bases, decision tres, e redes neurais artificiais; e outras tantas como estimação de densidade, classificação, regressão e clusterings. Portanto o que métodos de \glspl{bn} têm a oferecer? Segundo Heckerman \cite{heck95} podemos oferecer pelo menos quatro respostas, sendo elas:

\begin{enumerate}
	\item \glspl{bn} lidam com um conjunto incompleto de dados de maneira natural.

	\item \glspl{bn} permitem aprender sobre as relações causais. Aprender sobre tais relações são importantes por pelo menos duas razões: O processo é útil quando se está tentando entender sobre um dado problema de domínio, como por exemplo, durante uma análise de dados exploratória.  E mais, conhecimento de relações causais nos permitem fazer predições na presença de intervenções. Por exemplo, um analista de mercado pode querer saber se é lucrativo aumentar o investimento em determinada propaganda para aumentar as vendas de seu produto. Para responder esta pergunta o analista pode determinar se esta propaganda é a causa para o aumento de suas vendas, e em caso afirmativo, quanto. O uso de \glspl{bn} nos ajuda a responder tal pergunta até mesmo quando não há experimentos nos efeitos de tal propaganda.
	
	\item \glspl{bn} em conjunto com técnicas estatísticas bayesianas facilitam a combinação de conhecimento de domínio e dados. Qualquer um que tenha feito uma análise do mundo real sabe a importância de conhecimento prévio ou de domínio, em especial quando os dados são poucos ou caros. Pelo fato de alguns sistemas comerciais (i.e., sistemas especialistas) podem ser construídos a partir de conhecimentos prévios. \cglspl{bn} possuem uma semântica causal que permitem conhecimentos prévios serem representados de uma forma muito simples e natural. Além disto, \glspl{bn} encapsulam tais relações causais com suas probabilidades. Consequentemente, conhecimento prévio e dados podem ser combinados com técnicas bem estudadas da estatística Bayesiana.
	
	\item Métodos Bayesianos em conjunto com \glspl{bn} e outros tipos de modelos oferecem uma forma eficiente para evitar over fitting dos dados. Como veremos, não há necessidade de excluir parte dos dados do treinamento do aprendizado da rede. Usando técnicas Bayesianas, modelos podem ser "suavizados" de tal forma que todo dado disponível pode ser usado para o treinamento.

\end{enumerate}
\section{Probabilidade física versus probabilidade Bayesiana}
Para entender \glspl{bn} e as técnicas de aprendizado associadas, é importante entender a diferença entre a Probabilidade e Estatística padrão e a Bayesiana.

Resumidamente a probabilidade e padrão, também cohecida como probabilidade física é aquela que mede a probabilidade real do mundo. Como por exemplo o lançamento de moedas. É bem aceito que há 50\% de probabilidade de um lançamento de uma moeda não viciada dê cara, pois temos duas possibilidades e todas elas são igualmente prováveis. Outro exemplo é a probabilidade de tirar uma carta de paus dado um baralho padrão, dado que o baralho está embaralhado e que contém 13 cartas de cada naipe, a probabilidade da carta do topo do baralho ser de paus é 25\%. Isto é a probabilidade física, aquela probabilidade que todos estamos acostumados a lidar desde sempre.

No entanto, imagine agora que queremos determinar a probabilidade de tirar 

  %\chapter{Design Patterns}%


\section{Padrões de Comportamento}
\section{Padrões de Estrutura}
\section{Padrões de Construção}

  \input{tex/Arquitetura}
  \chapter{Como desenvolver um Plugin}%
Este documento serve de exemplo da utilização da classe \unbcic\ para escrever um texto cujo objetivo é apresentar os resultados de um trabalho. A sequência de ideias apresentada deve fluir claramente, de modo que o leitor consiga compreender os principais conceitos e resultados apresentados, bem como encontrar informações sobre conceitos secundários.

\url{http://www.escritacientifica.com/}
  % !TeX spellcheck = pt_BR
\chapter{Redes Bayesianas}%
Uma \gls{bn} provê uma representação compacta de distribuições de probabilidades grandes demais para lidar usando especificações tradicionais e provê um método sistemático e localizado para incorporar informação probabilística sobre uma dada situação.

Uma BN é um grafo acíclico direcionado (DAG) que representa uma função de distribuição de probabilidades conjunta de variáveis que modelam certo domínio de conhecimento. Ela é constituída de uma DAG, de variáveis aleatórias (também chamadas de nós da rede), arcos direcionados da variável pai para a variável filha e uma tabela de probabilidades condicionais (CPT) associada a cada variável.
\begin{figure}[H]
	\centering
	\includegraphics[width = 300px]{figuras/BN1}
	\caption[extraída do artigo do Laecio]{Exemplo family-out}
	\label{fig:familyBN}
\end{figure}

Nesse exemplo, suponhamos que se queira determinar se a família está em casa ou se ela saiu. Pelo grafo, pdoemos perceber que o fato de a luz da varanda estar acesa e de o cachorro estar fora de casa são indícios de que a família tenha saído. 

\section{Definição Formal}
Uma rede Bayesiana consiste em uma fatoração de uma distribuição de probabilidade e um DAG correspondente. Tais assertivas de independências condicionais podem ser inferidas diretamente da fatoração correspondente às 
  % inserir demais capítulos
  

  \postextual
  \bibliographystyle{plain}
  \bibliography{bibliografia}

\appendix
  %\begin{figure}[H]
	\centering
	\includegraphics[width = 200px]{figuras/singleton_pattern_uml_diagram}
	\caption {Diagrama UML do Design Pattern Singleton}
	\label{fig:dp_singleton}
\end{figure}
\begin{figure}[H]
	\centering
	\includegraphics[width = 350px]{figuras/factory_pattern_uml_diagram}
	\caption {Diagrama UML do Design Pattern Factory}
	\label{fig:dp_factory}
\end{figure}
\begin{figure}[H]
	\centering
	\includegraphics[width = 350px]{figuras/builder_pattern_uml_diagram}
	\caption {Diagrama UML do Design Pattern Builder}
	\label{fig:dp_builder}
\end{figure}
\begin{figure}[H]
	\centering
	\includegraphics[width = 350px]{figuras/prototype_pattern_uml_diagram}
	\caption {Diagrama UML do Design Pattern Prototype}
	\label{fig:dp_prototype}
\end{figure}


\begin{figure}[H]
	\centering
	\includegraphics[width = 350px]{figuras/adapter_pattern_uml_diagram}
	\caption {Diagrama UML do Design Pattern Adapter}
	\label{fig:dp_adapter}
\end{figure}
\begin{figure}[H]
	\centering
	\includegraphics[width = 350px]{figuras/bridge_pattern_uml_diagram}
	\caption {Diagrama UML do Design Pattern Bridge}
	\label{fig:dp_bridge}
\end{figure}
\begin{figure}[H]
	\centering
	\includegraphics[width = 200px]{figuras/composite_pattern_uml_diagram}
	\caption {Diagrama UML do Design Pattern Composite}
	\label{fig:dp_composite}
\end{figure}
\begin{figure}[H]
	\centering
	\includegraphics[width = 350px]{figuras/decorator_pattern_uml_diagram}
	\caption {Diagrama UML do Design Pattern Decorator}
	\label{fig:dp_decorator}
\end{figure}
\begin{figure}[H]
	\centering
	\includegraphics[width = 350px]{figuras/facade_pattern_uml_diagram}
	\caption {Diagrama UML do Design Pattern Facade}
	\label{fig:dp_facade}
\end{figure}
\begin{figure}[H]
	\centering
	\includegraphics[width = 350px]{figuras/proxy_pattern_uml_diagram}
	\caption {Diagrama UML do Design Pattern Proxy}
	\label{fig:dp_proxy}
\end{figure}


\begin{figure}[H]
	\centering
	\includegraphics[width = 350px]{figuras/chain_pattern_uml_diagram}
	\caption {Diagrama UML do Design Pattern Responsability}
	\label{fig:dp_responsability}
\end{figure}
\begin{figure}[H]
	\centering
	\includegraphics[width = 350px]{figuras/command_pattern_uml_diagram}
	\caption {Diagrama UML do Design Pattern Command}
	\label{fig:dp_command}
\end{figure}
\begin{figure}[H]
	\centering
	\includegraphics[width = 350px]{figuras/interpreter_pattern_uml_diagram}
	\caption {Diagrama UML do Design Pattern Interpreter}
	\label{fig:dp_interpreter}
\end{figure}
\begin{figure}[H]
	\centering
	\includegraphics[width = 350px]{figuras/iterator_pattern_uml_diagram}
	\caption {Diagrama UML do Design Pattern Iterator}
	\label{fig:dp_iterator}
\end{figure}
\begin{figure}[H]
	\centering
	\includegraphics[width = 350px]{figuras/mediator_pattern_uml_diagram}
	\caption {Diagrama UML do Design Pattern Mediator}
	\label{fig:dp_mediator}
\end{figure}
\begin{figure}[H]
	\centering
	\includegraphics[width = 350px]{figuras/null_pattern_uml_diagram}
	\caption {Diagrama UML do Design Pattern Null Object}
	\label{fig:dp_null_object}
\end{figure}


\begin{figure}[H]
	\centering
	\includegraphics[width = 350px]{figuras/observer_pattern_uml_diagram}
	\caption {Diagrama UML do Design Pattern Observer}
	\label{fig:dp_observer}
\end{figure}
\begin{figure}[H]
	\centering
	\includegraphics[width = 350px]{figuras/state_pattern_uml_diagram}
	\caption {Diagrama UML do Design Pattern State}
	\label{fig:dp_state}
\end{figure}

\begin{figure}[H]
	\centering
	\includegraphics[width = 350px]{figuras/template_pattern_uml_diagram}
	\caption {Diagrama UML do Design Pattern Template}
	\label{fig:dp_template}
\end{figure}
\begin{figure}[H]
	\centering
	\includegraphics[width = 350px]{figuras/visitor_pattern_uml_diagram}
	\caption {Diagrama UML do Design Pattern Visitor}
	\label{fig:dp_visitor}
\end{figure}



\end{document}
