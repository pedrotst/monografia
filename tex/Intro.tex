
\gls{bn} é um modelo gráfico para relações probabilísticas dado conjunto de variáveis. Nas últimas décadas, redes Bayesianas se tornaram representações populares para codificar conhecimento especialista incerto para sistemas especialistas \cite{heck95}. Mais recentemente, pesquisadores desenvolveram métodos de aprendizagem de redes Bayesianas a partir de dados. As técnicas desenvolvidas são novas e ainda em evolução, mas eles têm se mostrado muito eficientes para alguns problemas de análise de dados.

Existem diversos representações possíveis para analise de dados, entre elas, rule bases, decision tres, e redes neurais artificiais; e outras tantas como estimação de densidade, classificação, regressão e clusterings. Portanto o que métodos de \glspl{bn} têm a oferecer? Segundo Heckerman \cite{heck95} podemos oferecer pelo menos quatro respostas, sendo elas:

\begin{enumerate}
	\item \glspl{bn} lidam com um conjunto incompleto de dados de maneira natural.

	\item \glspl{bn} permitem aprender sobre as relações causais. Aprender sobre tais relações são importantes por pelo menos duas razões: O processo é útil quando se está tentando entender sobre um dado problema de domínio, como por exemplo, durante uma análise de dados exploratória.  E mais, conhecimento de relações causais nos permitem fazer predições na presença de intervenções. Por exemplo, um analista de mercado pode querer saber se é lucrativo aumentar o investimento em determinada propaganda para aumentar as vendas de seu produto. Para responder esta pergunta o analista pode determinar se esta propaganda é a causa para o aumento de suas vendas, e em caso afirmativo, quanto. O uso de \glspl{bn} nos ajuda a responder tal pergunta até mesmo quando não há experimentos nos efeitos de tal propaganda.
	
	\item \glspl{bn} em conjunto com técnicas estatísticas bayesianas facilitam a combinação de conhecimento de domínio e dados. Qualquer um que tenha feito uma análise do mundo real sabe a importância de conhecimento prévio ou de domínio, em especial quando os dados são poucos ou caros. Pelo fato de alguns sistemas comerciais (i.e., sistemas especialistas) podem ser construídos a partir de conhecimentos prévios. \cglspl{bn} possuem uma semântica causal que permitem conhecimentos prévios serem representados de uma forma muito simples e natural. Além disto, \glspl{bn} encapsulam tais relações causais com suas probabilidades. Consequentemente, conhecimento prévio e dados podem ser combinados com técnicas bem estudadas da estatística Bayesiana.
	
	\item Métodos Bayesianos em conjunto com \glspl{bn} e outros tipos de modelos oferecem uma forma eficiente para evitar over fitting dos dados. Como veremos, não há necessidade de excluir parte dos dados do treinamento do aprendizado da rede. Usando técnicas Bayesianas, modelos podem ser "suavizados" de tal forma que todo dado disponível pode ser usado para o treinamento.

\end{enumerate}
\section{Probabilidade física versus probabilidade Bayesiana}
Para entender \glspl{bn} e as técnicas de aprendizado associadas, é importante entender a diferença entre a Probabilidade e Estatística padrão e a Bayesiana.

Resumidamente a probabilidade e padrão, também cohecida como probabilidade física é aquela que mede a probabilidade real do mundo. Como por exemplo o lançamento de moedas. É bem aceito que há 50\% de probabilidade de um lançamento de uma moeda não viciada dê cara, pois temos duas possibilidades e todas elas são igualmente prováveis. Outro exemplo é a probabilidade de tirar uma carta de paus dado um baralho padrão, dado que o baralho está embaralhado e que contém 13 cartas de cada naipe, a probabilidade da carta do topo do baralho ser de paus é 25\%. Isto é a probabilidade física, aquela probabilidade que todos estamos acostumados a lidar desde sempre.

No entanto, imagine agora que queremos determinar a probabilidade de tirar 
