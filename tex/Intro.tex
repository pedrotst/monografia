% !TeX spellcheck = pt_BR
\chapter{Introdução}%
Redes Bayesianas é um modelo gráfico para relações probabilísticas dado conjunto de variáveis. Nas últimas décadas, redes Bayesianas se tornaram representações populares para codificar conhecimento especialista incerto para sistemas especialistas (Heckerman et al., 1995). Mais recentemente, pesquisadores desenvolveram métodos de aprendizagem de redes Bayesianas a partir de dados. As técnicas desenvolvidas são novas e ainda em evolução, mas eles têm se mostrado muito eficientes para alguns problemas de análise de dados.

Existem diversos representações possíveis para analise de dados, entre elas, rule bases, decision tress, e redes neurais artificiais; e outras tantas como estimação de densidade, classificação, regressão e clusterings. Portanto o que métodos de redes Bayesianas têm a oferecer? Podemos oferecer pelo menos quatro respostas.

Primeiramente, redes Bayesianas lidam com um conjunto incompleto de dados de maneira natural.

Em segundo lugar, BNs permitem aprender sobre as relações causais. Aprender sobre tais relações são importantes por pelo menos duas razões: O processo é útil quando se está tentando entender sobre um dado problema de domínio, como por exemplo, durante uma análise de dados exploratória. 