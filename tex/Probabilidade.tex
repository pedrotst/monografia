\section{Probabilidade física versus probabilidade Bayesiana}
Para entender \glspl{bn} e as técnicas de aprendizado associadas, é importante entender a diferença entre a Probabilidade e Estatística padrão e a Bayesiana.

%Resumidamente a probabilidade bayesiana de um evento x é o grau de crença de uma pessoa naquele evento. Enquanto a probabilidade clássica é é uma propriedade física do mundo (por exemplo a probabilidade de um lançamento de moeda dar cara), a probabilidade Bayesiana é uma propriedade da pessoa que diz a probabilidade (por exemplo seu grau de crença do lançamento dar cara).

A Probabilidade física, também conhecida como probabilidade frequentista, foi advogada pelo matemático John Venn \cite{ven66} no século XIX, identificando probabilidade com frequências dos eventos no longo prazo. A reclamação mais óbvia que surge neste modelo é que as frequências no curto prazo obviamente não casam com as calculadas, por exemplo, se jogarmos a moeda apenas uma vez, certamente concluiríamos que a probabilidade de cara é ou 1 ou 0.

Uma alternativa ao conceito de probabilidade física é pensar nas probabilidades como nosso grau de crença subjetivo. Esta visão foi expressa por Thomas Bayes \cite{bayes63} e Pierre Simon de Laplace \cite{laplace12} a duzentos anos atrás. Esta é uma visão mais. Esta é uma visão mais geral da probabilidade pois leva em conta que temos crenças subjetivas em um grande variedade de proposições, muitas das quais não estão claramente ligadas a um processo físico.  Por exemplo, a maioria de nós acreditamos na Hipótese de Copérnico de que a terra orbita em torno do sol, mas isto é baseado em evidencia não obviamente da mesma forma que um processo de amostragem. Isto é, ninguém é capaz de gerar sistemas solares repetidamente e observar a frequência com a qual os planetas giram em torno do sol. Seja como for Bayesianistas estão preparados para conversar sobre a probabilidade da tese de Copérnico ser verdadeira e ainda relacionar as relações de evidências a favor e contra ela. 

Bayesiansmo pode ser visto como uma generalização da probabilidade física. Para isto adotamos o que David Lewis apelidou de Principal Principle \cite{lewis80}: sempre que se aprender uma probabilidade física de uma amostragem é r, atualize sua probabilidade subjetiva para aquela amostragem r. Basicamente isto é senso comum: pense na probabilidade de um colega raspar a cabeça como 0.01, mas se você aprende que ele faz isso se e somente se um dado justo for lançado e der 2, você revisará sua opinião de acordo.

Tendo explicado como as probabilidades Físicas e Bayesianas são compatíveis podemos seguir para a definição dos axiomas da probabilidade. 

\section{Axiomas da Probabilidade}
Seja $U$ o universo de possíveis evento. Os Axiomas de Kolmogorov \cite{kolmogorov33} são:

\newtheorem{kolg1}{Axiom}[chapter]
\newtheorem{kolg2}[kolg1]{Axiom}
\newtheorem{kolg3}[kolg1]{Axiom}
\begin{kolg1}
	$P(U) = 1$
\end{kolg1}

Onde $P$ é uma massa de probabilidade, somando ou integrando em 1, é distribuído sobre $U$, uniformemente ou não. 

\begin{kolg2}
	$for\ all\ X \subseteq U, P(X) \ge 0$
\end{kolg2}


\begin{kolg3}
	$for\ all\ X, Y \subseteq U,\ if\ X \cap Y = \emptyset,\ then\ P(X \cup Y) = P(X) + P(Y)$
\end{kolg3}
