No cotidiano do ser humano é muito comum raciocinar e tomar decisões sob condições de incerteza. Isto é tão visível e profundo para algumas pessoas que no século XVIII Bishop Butler declarou "probabilidade é o guia da vida".

Como um exemplo do quanto probabilidade é importante tome a medicina. Para um especialista médico determinar a doença de uma pessoa com base nos sintomas observados (evidências) é preciso que ele leve em conta a probabilidade daquele sintoma refletir esta ou aquela doença, pois a doença ocasiona determinado sintoma apenas com alguma probabilidade, mas não com certeza.

Podemos ainda citar vários outros exemplos, como na economia, teoria dos jogos, genética, previsão do tempo.

Pensando nisso é necessário um modelo robusto para lidar com tantas incertezas, para tanto escolhemos as \glspl{bn}.


\section{Objetivos}
O objetivo deste trabalho é apresentar os conhecimentos adquiridos durante este semestre na disciplina de Estudos Em Inteligência Artificial com o professor Ladeira da \gls{unb}, com a colaboração do Doutorando Shou Matsumoto da \gls{gmu}, através de aulas pelo skype. E também firmar bases sólidas sobre os conhecimentos que serão necessários na concepção da verdadeira monografia, isto é, o trabalho de conclusão de curso.


\section{Estrutura da Monografia}
Este documento é organizado da seguinte maneira: no capítulo 2 explicamos a visão probabilística necessária, isto é, firmamos a base do trabalho que se segue, como todos axiomas e definições da probabilidade necessários. No capítulo 3 formalizamos o conceito de \glspl{bn} e algumas de suas propriedades. No capítulo 4 discutimos como construí-las a partir de um conjunto de dados. No capítulo 5 explicamos o framework do UnBBayes, como ele funciona e como extendê-lo. O capítulo 6 discutimos alguns dos Design Patterns mais importantes. No capítulo 7 resumimos o trabalho do semestre e oferecemos uma prévia dos trabalhos a serem feitos nos próximos semestres.


