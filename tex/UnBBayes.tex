%\chapter{UnBBayes}%
UnBBayes é uma aplicação open-source feita em Java\texttrademark desenvolvido pelo \gls{gia} do Departamento de Ciência da Computação da \gls{unb} no Brasil e provê um framework para construir modelos gráficos probabilísticos e realizar raciocínios plausíveis. Ele apresenta uma \gls{gui}, \gls{api} e ainda suporte a plug-ins para extensões não previstas.

%\cite{livro_do shou}%
 descreve os três objetivos principais das mais novas versões do UnBBayes, são eles:
\begin{itemize}
	\item Ser uma plataforma para a disseminação dos conceitos e utilidade do raciocinio probabilistico;
	\item Ser uma ferramenta visual fácil de usar e configurar;
	\item Fornecer extensibilidade.
\end{itemize}
O Primeiro objetivo é atingido com a implementação do estado da arte de \gls{bn} e um algoritmo de inferência padrão baseado no algoritmo de Junction Trees. O segundo através de uma a implementação de GUI intuitiva 

\section{Plugins}

\subsection{Como desenvolver plugins}