Durante esse semestre de 2015/2 foram iniciados os estudos em conjunto com o Guilherme Carvalho para que desenvolvamos nosso TCC (independentemente) nos semestres que estão por vir.

Este capítulo tem por objetivo oferecer um resumo das atividades desenvolvidas no decorrer deste semestre e delinear as atividades a serem feitas nos semestres seguintes. 

\section{Atividades Realizadas}
No decorrer deste semestre o professor Shou Matsumoto, douturando da \gls{gmu} ministrou aulas semanais às segundas via Skype sobre os conteúdos de: 

\begin{enumerate}
	\item Preparação do ambiente UnBBayes. Com o objetivo de instalar o Maven e o Subversion no eclipse e deixar o ambiente pronto para trabalhar, consertando todos possíveis problemas de dependencias;
	\item Design Pattern e Anti-Patterns;
	\item Arquitetura do UnBBayes;
	\item Como desenvolver um Plug-in para o UnBBayes;
	\item Redes Bayesianas;
	\item MEBN;
	\item PR-OWL;
	\item Diagramas de Influência.
\end{enumerate}

No decorrer do semestre também fui escalado para implementar ou consertar tarefas de usabilidade no UnBBayes. Foram elas:
\begin{enumerate}
	\item Implementar crtl+c e crtl+v de nós para a extensão MEBN;
	\item Implementar deleção de múltiplos findings de MEBN;
	\item Refatorar a \gls{cpt} para prover uma melhor visualização de tabelas muito grandes.
\end{enumerate}

Para a primeira tarefa tentei refatorar o crtl+c crtl+v já existente no core para que cada nó saiba como se copiar, entretanto tive problemas com as chamadas de funções, pois a funcionalidade de cada nó não estava encapsulada o suficiente para fazê-lo, então por motivo de falta de tempo não pude ir adiante.

A terceira tarefa foi implementada através da não-scrolagem da primeira coluna.

\section{Atividades que serão realizadas}
Ficou decidido que nos semestres subsequentes será implementado um novo módulo de Incremental Learning, utilizando-se as mais novas tecnicas disponíveis. 
E com isto analisar dados de Bactérias disponibilizados pelo professor Ladeira.

Nos primeiros meses iremos levantar um estudo do estado da arte sobre learning e incremental learning, analisar as necessidades dos dados que serão analisados, isto é, os dados tem valores faltantes? São independentes entre si? Possuem variáveis contínuas? Como visto no \autoref{ch:aprendizado}, todas estas perguntas influenciam na escolha do algoritmo e nas técnicas que deveremos estudar nos próximos meses.

O cronograma atualizado a ser realizado nos próximos semestres pode ser encontrado em \cite{cronograma}. E não foi disponibilizado aqui por motivos de espaço.

Finalmente podemos concluir que este semestre foi proveitoso o suficiente para iniciar nossa compreensão em \gls{bn}, learning, a ferramenta UnBBayes. E que estamos preparados para concluir uma tarefa robusta e auto-contida até o final de 2016.